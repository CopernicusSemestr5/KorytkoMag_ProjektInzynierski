\chapter{RET}
	\section{Prezentacja urządzenia}
		Na rysunku \ref{fig:Obrazek_RET1} przedstawiono miejsce w urządzeniu RET, w którym należy umieścić antenę, po to, aby 
		zmienić kąt głównej wiązki sygnału przez nią promieniowanego. Na grafice \ref{fig:Obrazek_KatNachyleniaAnteny} 
		przedstawiono lokalizację omawianego kąta głównego w odniesieniu do położenia użytkownika telefonu komórkowego oraz 
		stacji nadawaczej.
		W celu przeprowadzania automatycznych testów sprzętowych w zaawansowanych laboratoriach, należy zamiast fizycznej anteny, umieścić jej atrapę co można zaobserwować
		na grafice \ref{fig:Obrazek_RET2}.
		\begin{figure}[h!]
			\centering
			\includegraphics[scale=0.4]{Obrazki/RET_1.png}
			\caption{RET - Miejsce na antenę.
				\newline(Zdjęcie własne)}
			\label{fig:Obrazek_RET1}	
		\end{figure}

		\begin{figure}[h!]
			\centering
			\includegraphics[scale=0.4]{Obrazki/RET_2.png}
			\caption{RET - podłączonym kablem RS-485 oraz włożoną atrapą anteny.
				\newline(Zdjęcie własne)}
			\label{fig:Obrazek_RET2}
		\end{figure}

	\section{Zmiana szerokości głównej wiązki fali elektromagnetycznej}
		\begin{figure}[h!]
			\centering
			\includegraphics[scale=1.0]{Obrazki/KatNachyleniaAnteny.png}
			\caption{Rysunek przedstawiający kąt modyfikowany przy pomocy RET-a.
				\newline(Opracowanie własne)}
			\label{fig:Obrazek_KatNachyleniaAnteny}	
		\end{figure}
		
		W celu zobrazowania zmieniającej się szerokości wiązki promieni sygnału radiowego w osi poziomej\cite{BEAMWIDTH_1}
		podczas modyfikacji elektrycznego kąta z wartości 0-ra stopni (rysunek \ref{fig:Obrazek_KatNachyleniaAnteny_0_Stopni})
		na wartość 40-tu stopni (rysunek \ref{fig:Obrazek_KatNachyleniaAnteny_40_Stopni}), dzięki programowi Radio Mobile\cite{RADIO_MOBILE_PAGE_1} wygenerowano poglądowe diagramy.
		Podczas symulacji użyto antenę Yagi, która aktualnie nie jest stosowana w technologii mobilnej z racji wspieranych
		przez nią częstotliwości, gdyż są one znacznie niższe aniżeli te wymagane przez WCDMA, LTE czy \textit{5G}, 
		aczkolwiek bardzo dobrze odzwierciedla działanie tych rzeczywistych ze względu na swoją charakterystykę kierunkową.
		\newline
		Antenę umieszczono przy WSIZ Copernicus. \newline
		Pomarańczowym kolorem przedstawiono charakterystykę zysku promieniowania anteny, które w przypadku kąta 0 stopni jest wydłużone oraz węższe,
		co pozwala pokryć śladowym zasięgiem nawet dalekie obszary, lecz do terenów bliżej zlokolizowanych dostarczony
		jest słabszy sygnał względem tego, który można otrzymać, zmieniając elektryczny kąt anteny na 40 stopni.
		Dzięki RET-owi, można skoncentrować wiązkę na mniejszym obszarze, oferując znacznie wyższe prędkości transmisji danych.

		\begin{figure}[h!]
			\centering
			\includegraphics[scale=0.5]{Obrazki/Antenna_Yagi_Angle_0.png}
			\caption{Pokrycie obszaru sygnałem radiowym dzięki antenie Yagi, kąt elektryczny 0 stopni.
				\newline(Opracowanie własne przy pomocy programu Radio Mobile)}
			\label{fig:Obrazek_KatNachyleniaAnteny_0_Stopni}
		\end{figure}

		\begin{figure}[h!]
			\centering
			\includegraphics[scale=0.5]{Obrazki/Antenna_Yagi_Angle_40.png}
			\caption{Pokrycie obszaru sygnałem radiowym dzięki antenie Yagi, kąt elektryczny 40 stopni.
				\newline(Opracowanie własne przy pomocy programu Radio Mobile)}
			\label{fig:Obrazek_KatNachyleniaAnteny_40_Stopni}
		\end{figure}
