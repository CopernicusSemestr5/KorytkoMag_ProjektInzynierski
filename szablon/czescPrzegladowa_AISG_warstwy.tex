\chapter{Protokół AISG 2.0}
	Symulator sterownika ma za zadanie odebrać od użytkownika polecenie w postaci komendy wieloargumentowej, 
	która zawierała będzie nazwę procedury rozpoznawanej przez odpowiednie warstwy protokołu komunikacyjnego \textit{AISG v2.0}
	realizującego podzbiór \textit{modelu OSI}\cite{ModelOSI}.
	
	\section{Warstwy}
		\subsection{1-sza - Fizyczna}
		Realizacja pracy w kierunku emulacji fizycznego połączenia do urządzenia niesie ze sobą pewne zmiany na tej warstawie.\newline
		Należy pominąć obszar mechaniczny i elektryczny\cite{ETSI-3GPP-0}, a skupić się na obszarze funkcjonalnym oraz proceduralnym.
		\subsection{2-ga - Łącza danych}
		Rolą tej warstwy jest\cite{ETSI-3GPP-2}:
		\begin{itemize}
			\item Enkapsulacja\cite{ENKAPSULACJA} ramki \textit{HDLC Body} do ramki HDLC\cite{WIKI_ENG_HDLC};
			\begin{itemize}
				\item Dodanie \textit{bitów} startu i stopu;
				\item Obliczenie oraz walidacja sumy CRC\cite{CRC};
			\end{itemize}
			\item Budowa ramki typu XID podczas procedur negocjacji:
			\begin{itemize}
				\item Rozmiaru ramki;
				\item Unikalnego identyfikatora urządzenia, na które składa się numer seryjny oraz kod producenta;
				\item Wersji \textit{3GPP} oraz AISG urządzenia podrzędnego;
				\item Adresu;
			\end{itemize}
			\item Budowa ramki typu U w celu ustanowienia normalnego trybu komunikacji;
		\end{itemize}
		\subsection{7-ma - Aplikacji}
			\begin{itemize}
				\item Budowa ramki typu I;
				\item Rozpoznawanie wysokopoziomowej komendy kalibruj\cite{ETSI-3GPP-7};
			\end{itemize}
