\section{HDLC}
Z języka angielskiego High-Level Data Link Control. Protokół warstwy łącza danych modelu OSI. 
Standard HDLC opisuje norma ISO, lecz szeroko stosuje się także implementację CISCO.
HDLC jest stosowany w technologii WAN, obsługuje zarówno połączenia dwupunktowe, jak i wielopunktowe. 
Jest protokołem o orientacji bitowej oraz jest przezroczysty informacyjnie. \autocite{WIKI_HDLC}
W przypadku jeśli przesyłana wartość jest wielobajtowa, zastosowane jest podejście cienkokońcowości.
\footnote{\label{Little endian} Little endian - inaczej cienkokońcowość, forma zapisu danych o rozmiarze większym niż jeden bajt w której najmniej znaczący bajt umieszczony jest jako pierwszy.}
\subsection{Struktura ramki}
\begin{enumerate}
    \item Flaga startu - 0x7E;
    \item Adres stacji docelowej;
    \item Sterowanie - określa typ ramki oraz jej parametry w zależności od typu;
    \item Dane;
    \item Suma kontrolna FCS ( dwubajtowa ) - na przykład CRC-16;
    \item Flaga stopu - 0x7E;
\end{enumerate}
\subsection{Typy ramek}
\begin{itemize}
    \item Ramka I - Informacyjna;
    \item Ramka U - Nienumerowana;
    \item Ramka S - Nadzorująca;
    \item Ramka XID - Identyfikująca urządzenia;
\end{itemize}
% WIKI