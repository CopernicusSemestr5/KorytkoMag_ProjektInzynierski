% \chapter{Analiza nawiązanej komunikacji}
\section{Skanowanie urządzeń}
Parametrem tej komendy jest adres portu 5555, z którego korzysta sterownik
do nawiązania połączenia typu tcp na adresie 127.0.0.1 wraz z symulatorem urządzenia,
przy zastosowaniu wzorca Żadanie-Odpowiedź. Jest to wzorzec, który zakłada otrzymanie odpowiedzi na każdą wysłaną wiadomość, zanim kolejna zostanie nadana. 
Opisany mechanizm komunikacji aplikowalny jest również do wszystkich poniższych komend.

Analiza poszczególnych wartości wiadomości wychodzącej (rysunek \ref{lst:WykonanieProgramu}, linijki {15, 16}):
\begin{enumerate}
    \item ADDR = 0xFF --- brak adresu a więc skanowanie przebiega wobec wszystkich urządzeń na linii bez adresu; 
    które nie są zaadresowane, a więc jest to wiadomość typu \textit{broadcast};
    \item CTRL = 0xBF --- charakterystyczny dla ramki XID;
    \item FI = 0x81 --- charakterystyczna dla ramki XID oraz żądania przypisania adresu;
    \item GI = 0xF0 --- charakterystyczna dla ramki XID oraz żadania przypisania adresu;
    \item GL = 0x08 --- liczba bajtów która pojawi się od kolejnej pozycji aż do pierwszego batja sumy CRC;
\end{enumerate}

Analiza poszczególnych wartości wiadomości przychodzącej 
(rysunek \ref{lst:WykonanieProgramu}, linijka 17} oraz \ref{fig:DiagramSequence_AddressAssignment_SecondDeviceScan}):
\begin{enumerate}
    \item ADDR = 0x00 --- urządzenie jest jest w stanie niezaadresowanym, a więc domyślna wartość to 0x00
    które nie są zaadresowane, a więc jest to wiadomość typu \textit{broadcast};
    \item CTRL = 0xBF --- charakterystyczny dla ramki XID;
    \item FI = 0x81 --- charakterystyczny dla ramki XID oraz żądania przypisania adresu;
    \item GI = 0xF0 --- charakterystyczny dla ramki XID oraz żadania przypisania adresu;
    \item GL = 0x12 --- liczba bajtów która pojawi się od kolejnej pozycji aż sumy CRC;
    \item Pierwszy parametr HDLC:
    \begin{enumerate}
        \item PI = 0x01 --- unikalny identyfikator urządzenia;
        \item PL = 0x09;
        \item PV = \{ 0x4E, 0x4B, 0x34, 0x36, 0x35, 0x30, 0x30, 0x30, 0x30 \};
    \end{enumerate}
    \item Drugi parametr HDLC:
    \begin{enumerate}
        \item PI = 0x04 --- typ urządzenia;
        \item PL = 0x01;
        \item PV = 0x01 --- pojedynczy RET;
    \end{enumerate}
    \item Trzeci parametr HDLC:
    \begin{enumerate}
        \item PI = 0x06 --- kod producenta;
        \item PL = 0x02;
        \item PV = \{ 0x4E, 0x4B \};
    \end{enumerate}
\end{enumerate}


