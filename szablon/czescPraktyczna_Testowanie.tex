\chapter{Testowanie oprogramowania}
    Projekt zrealizowano zgodnie z zasadą TDD, którą budują trzy główne prawa.
    \newline
    Pierwsze prawo - nie można zacząć pisać kodu produkcyjnego do momentu napisania niespełnianego testu jednostkowego.
    \newline
    Drugie prawo - Nie można napisać więcej testów jednostkowych, które są wystarczające do niespełnienia testu, a brak kompilacji jest jednocześnie nieudanym testem.
    \newline
    Trzecie prawo - Nie można pisać większej ilości kodu produkcyjnego, niż wystarczy do spełnienia obecnie niespełnianego testu\cite{martin2014czysty}.
    \newline
    Dzięki powyższym postulatom, projekt nie zawiera kodu nieużywanego, który wymagałby dodatkowego narzutu czasu podczas jego czytania. Prawdą jest to, że lepszy jest kod,
    którego nie ma, aniżeli ten który jest, aczkolwiek jego działanie jest niewiadome. Obszerna regresja na wielu poziomach abstrakcji pozwoliła na ciągłe usprawnianie zarówno architektury
    jak i dostarczonych funkcjonalności bez obaw wprowadzenia błędów. Podczas konstrukcji przypadków testowych zwrócono uwagę na to, że odpowiednie zdefiniowanie: 
    argumentów wejściowych funkcji/klasy, wartości zwracanych przez funkcję/metodę, pozwoli zachować jedno ciało funkcji testowych. Podejście to nazywa się testowaniem parametrycznym.

%%%%%%%%%%%%%%%%%%%%%%%%%%%%%%%%%%%%%%%%%%%%%%%%%%%%%%%%%%%%%%%%%%%%%%%%%%%%%%%%%%%%%%%%%%%%%%%%%%%%%%%%%%%%
    
\section{Testy jednostkowe}
    \lstinputlisting[
    language=bash,
    basicstyle=\tiny,
    label=lst:TestyJednostkowe,
    caption=Efekt uruchomienia testów jednostkowych]
    {LogZWykonania/LogsFromUT.log}
    
    Testy jednostkowe zwane są często testami komponentów, czy też testami modułowymi. Ich głównym celem jest znalezienie błędów 
    w implementacji danej jednostki / komponentu\cite{Testowanie}.
    
    Uruchomienie testów odbywa się przy pomocy komendy ,,bash runUT.sh'', wywoływanej z poziomu głównego katalogu projektu.
    
    \subsection{Testowanie bazy danych}
    \label{sec:TestowanieBazyDanych}
        Przeprowadzone testy (listing \ref{lst:TestyJednostkowe}- linijki [7; 22]):
        \begin{enumerate}
            \item Pobierz obiekt przy pomocy niestniejącego klucza.
            \item Dodaj nowy obiekt przy pomocy klucza.
            \item Wielokrotnie dodaj nowy obiekt przy pomocy klucza.
            \item Dodaj nowy obiekt bez pomocy klucza.
            \item Wielokrotnie dodaj nowy obiekt bez pomocy klucza.
            \item Skasuj obiekt przy pomocy klucza.
            \item Wielokrotnie skasuj obiekt przy pomocy klucza.
        \end{enumerate}
    \subsection{Testowanie budowania pełnej ramki AISG}
        Przeprowadzony test (listing \ref{lst:TestyJednostkowe}- linijki [24; 27]) sprawdza czy konstruktor klasy ,,HDLCFrame'' poprawnie
        dokona enkapsulacji ciała ramki, wzbogacając ją o flagę startu, stopu oraz sumę kontrolną. Dzięki obiektowemu podejściu do tworzenia kodu źródłowego, 
        brak jest konieczności wykonania testów dla każdej z budowanych ramek, ponieważ dokonano tego klasie testowej dedykowanej do weryfikacji poprawności budowania ciała ramki. 
    \subsection{Testowanie parsera danych wejściowych użytkownika}
        Przeprowadzone testy (listing \ref{lst:TestyJednostkowe}- linijki [29; 36]):
        \begin{enumerate}
            \item Wprowadź poprawnie prawidłową komendę
            \item Wprowadź niepoprawnie prawidłową komendę, brak zachowania odpowiedniej wielkość liter
            \item Wprowadź nieprawidłową komendę
        \end{enumerate}
    \subsection{Testowanie fabryki tworzącej ciało ramki AISG}
        Przeprowadzone testy (listing \ref{lst:TestyJednostkowe}- linijki [38; 57]) pokrywają zbudowanie każdej z ramek
        wspomnianej w rozdziale ,,Analiza nawiązanej komunikacji''[\ref{ch:AnalizaNawiazanejKomunikacji}].
    \subsection{Testowanie interpretera ramki HDLC}
        Przeprowadzone testy (listing \ref{lst:TestyJednostkowe}- linijki [59; 76]) pokrywają rozpoznanie każdej z ramek 
        wspomnianej w rozdziale ,,Analiza nawiązanej komunikacji''[\ref{ch:AnalizaNawiazanejKomunikacji}], wprowadzonej w postaci łańcucha znaków.

%%%%%%%%%%%%%%%%%%%%%%%%%%%%%%%%%%%%%%%%%%%%%%%%%%%%%%%%%%%%%%%%%%%%%%%%%%%%%%%%%%%%%%%%%%%%%%%%%%%%%%%%%%%%

\section{Testy integracyjne}
    \lstinputlisting[
    language=bash,
    basicstyle=\tiny,
    label=lst:TestyIntegracyjne,
    caption=Testy integracyjne]
    {LogZWykonania/LogsFromMT.log}
    
    Kolejnym poziomem są testy integracyjne, sprawdzają czy komponenty poprawnie współdziałają ze sobą\cite{Testowanie}. 
    Uruchomienie testów odbywa się przy pomocy komendy ,,bash runMT.sh'', wywoływanej z poziomu głównego katalogu projektu.
    
    \subsection{Integracja bazy danych z interfejsem użytkownika}
    Przeprowadzone testy (listing \ref{lst:TestyIntegracyjne}- linijki [7; 18]) pokrywają przypadki testowe zdefiniowane w rozdziale ,,Testowanie bazy danych''
    \ref{sec:TestowanieBazyDanych}.
    Różnicą jest sposób przekazania danych wejściowych. W poprzednim rozdziale w ciele funkcji testowej, wywoływano poszczególne funkcje interfejsu bazy danych.
    Testy integracyjne dokonują modyfikacji bazy danych, korzystając z funkcji obiektu klasy ,,CMenu''. Przy jego pomocy, końcowy użytkownik wchodzi w interakcję ze sterownikiem  
    przy pomocy interfejsu konsoli. Dzięki odpowiedniej definicji metod klasy ,,CMenu'', na poziomie klasy testowej, pominięto fazę komunikacji przy pomocy konsoli.
    Zastąpiono go wywołaniem funkcji, która jako argument przyjmuje obiekt typu std::vector<std::string>, co można zaobserwować na listingu
    \ref{lst:TestyIntegracyjne_UI_Database}. Przechowuje on podzieloną na tokeny rzeczywistą komendę, która ma za zadanie zmianę stanu bazy danych.
    \lstinputlisting[
        language=bash,
        basicstyle=\tiny,
        caption=Ciało przykładowego testu integracyjnego interfejsu użytkownika wraz z bazą danych,
        label=lst:TestyIntegracyjne_UI_Database]
        {Kod_Zrodlowy/TestyIntegracyjne_UI_Database.cpp}

    \subsection{Integracja interfejsu użytkownika z kontrolerem komend AISG}
    Przeprowadzone testy (listing \ref{lst:TestyIntegracyjne}- linijki [20; 37]) weryfikują działanie komend zdefiniowanych w rozdziale ,,Wymagania funkcjonalne''
    \ref{ch:WymaganiaFunkcjonalne}.
    Podobnie jak w przypadku testowania integracji bazy danych z interfejsem użytkownika. Dane wejściowe testu są obiektem typu std::vector<std::string>, który przechowuje
    rzeczywistą komendę podzieloną na tokeny. Ma ona za zadanie wykonanie jednej z komend rozpoznawaną przez warstwę fizyczną, łącza danych czy aplikacyjną.
    Pomyślność integracji komponentów weryfikowana jest przy pomocy kodu rezultatu, zwracanego przez komendę ,,execute'' (rysunek \ref{lst:TestyIntegracyjne_UI_Ctrl})
    należącą do klasy kontrolera komend AISG.
    \lstinputlisting[
        language=bash,
        basicstyle=\tiny,
        caption=Ciało przykładowego testu integracyjnego interfejsu użytkownika oraz kontrolera komend AISG wraz z definicją przykładowych parametrów testu,
        label=lst:TestyIntegracyjne_UI_Ctrl]
        {Kod_Zrodlowy/TestyIntegracyjne_UI_Ctrl.cpp}
\newpage

    \subsection{Integracja interfejsu użytkownika, kontrolera komend AISG oraz fabryki ramek HDLC}
    Przeprowadzone testy (listing \ref{lst:TestyIntegracyjne}- linijki [39; 50]) weryfikują działanie komend zdefiniowanych w rozdziale ,,Wymagania funkcjonalne''
    \ref{ch:WymaganiaFunkcjonalne}.
    Na wejściu testu przekazywany jest obiekt typu std::vector<std::string> (rysunek \ref{lst:TestyIntegracyjne_UI_Ctrl_Frames}), który przechowuje rzeczywistą komendę podzieloną na tokeny. Rezultat wykonania
    weryfikowany jest z ramką HDLC budowaną przy pomocy fabryki użytej w kodzie produkcyjnym.
    \lstinputlisting[
        language=bash,
        basicstyle=\tiny,
        caption=Ciało przykładowego testu integracyjnego interfejsu użytkownika oraz kontrolera komend AISG wraz z definicją przykładowych parametrów testu,
        label=lst:TestyIntegracyjne_UI_Ctrl_Frames]
        {Kod_Zrodlowy/TestyIntegracyjne_UI_Ctrl_Frames.cpp}
\section{Manualne testy systemowe}
Dzięki zaznajomieniu się z wieloma wzorcami projektowymi jak fabryka, komenda, 
budowniczy czy nakładanie obostrzeń na program dzięki listowaniu obsługiwanych komend, wspomniany wcześniej symulator urządzenia powstał
przy minimalnym nakładzie pracy, co jest bardzo dużą oszczędnością. Świadczy to o sukcesie płynącego z 
dobrze wykonanego projektu architektury oprogramowania. Kolejną zaletą utworzenia symulatora urządzenia jest umożliwienie przeprowadzenia manualnych testów komponentowych, 
realizując regułę czarnej skrzynki. Symulator urządzenia wraz z sterownikiem komunikuje się przy pomocy biblioteki ZeroMQ.

%%%%%%%%%%%%%%%%%%%%%%%%%%%%%%%%%%%%%%%%%%%%%%%%%%%%%%%%%%%%%%%%%%%%%%%%%%%%%%%%%%%%%%%%%%%%%%%%%%%%%%%%%%%%
\newpage
\section{Automatyczne testy systemowe}
Implementacja automatycznych testów systemowych osiągalna będzie w momencie wystawienia przez programistę interfejsu aplikacji. Dzięki kompleksowemu pokryciu utworzonego
kodu źródłowego testami na niższych poziomach, konfiguracja środowiska automatyzacji oraz tworzenie scenariuszy wykonania nie przynosi zbyt dużego zysku. 
Dopiero w momencie uruchomienia więcej niż jednej instancji sterownika, komunikującej się z wieloma fizycznymi bądź emulowanymi urządzeniami, faza automatycznych testów
systemowych przyniesie wymierne korzyści. Z racji tego, że skalowalność w górę utworzonego systemu planowana jest na dalsze etapy implementacji, zdefiniowano zamiar
wzbogacenia regresji o ten poziom testowania.
