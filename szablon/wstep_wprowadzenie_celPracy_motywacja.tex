\tableofcontents

\chapter{Wstęp}
\section{Wprowadzenie}
	Stacja nadawcza to zintegrowany system składający się z modułu systemowego, modułu rozszerzeniowego, radia oraz fizycznej anteny. 
	Ma ona na celu dostarczenie jak największej liczbie ludzi sygnału telekomunikacyjnego w celu nawiązania połączenia głosowego, 
	wysłania smsa czy skorzystania ze skrzynki mailowej. Sęk w tym, że stacja nadawcza może być umieszczona w jednym miejscu, 
	natomiast odbiorcy mogą przemieszczać się co niesie ze sobą problem efektywnego pokrycia obszaru zasięgiem sieci operatora.
	W celu modyfikacji obszaru aktualnie pokrytego zasięgiem, anteny dipolowe usprawniane są o dodatkowe urządzenie 
	które nosi nazwę RET, które jest szeroko stosowane w technologiach GSM, WCDMA, LTE.

\section{Cel pracy}
	Jako cel obrałem zaznajomienie się z protokołem komunikacyjnym AISG 2.0, którego użycie możemy zaobserwować na drodze pomiędzy 
	radio modułem a wzmacniaczem antenowym czy RET-em. 
	Jest to krok obowiązkowy przed rozpoczęciem zapoznawania się z AISG 3.0.
	Implementacja protokołu będzie wymagała wysokich umiejętności programowania w języku C++, 
	tworzenia testów, wdrażania wzorców projektowych oraz wiedzy z zakresu systemu kontroli wersji czy inżynierii oprogramowania.

\section{Motywacja}
	W przyszłości podłączenie RET-a do komputera (pomijając radio moduł) przy pomocy adaptera RS-232 -> USB, w celu skrócenia czasu testowania urządzenia.
