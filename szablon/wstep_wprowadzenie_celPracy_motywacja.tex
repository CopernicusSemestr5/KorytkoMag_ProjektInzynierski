\chapter{Wstęp}
\section{Wprowadzenie}
	Stacja nadawcza to zintegrowany system składający się z modułu systemowego, modułu rozszerzeniowego, radia oraz fizycznej anteny. 
	Ma ona na celu dostarczenie jak największej liczbie ludzi sygnału telekomunikacyjnego w celu nawiązania połączenia głosowego, 
	wysłania wiadomości tekstowej czy skorzystania ze skrzynki mailowej. Głównym problemem jest to, że stacja nadawcza może być umieszczona w jednym miejscu, 
	natomiast odbiorcy, mogą przemieszczać się, co niesie ze sobą problem efektywnego pokrycia obszaru zasięgiem sieci operatora.
	W celu modyfikacji charakterystyki sygnału, anteny dipolowe usprawniane są o dodatkowe urządzenie o nazwie \textit{RET}, które jest szeroko stosowane w technologiach 
	\textit{GSM}, \textit{WCDMA} czy \textit{LTE}.

\section{Cel pracy}
	Jako cel obrano zaznajomienie się z protokołem komunikacyjnym AISG 2.0, którego użycie można zaobserwować na drodze pomiędzy 
	radio modułem a wzmacniaczem antenowym czy RET-em. Jest to krok obowiązkowy przed rozpoczęciem zapoznawania się z AISG 3.0.
	Implementacja protokołu będzie wymagała wysokich umiejętności programowania w języku \textit{C++}, 
	tworzenia testów, wdrażania wzorców projektowych oraz wiedzy z zakresu systemu kontroli wersji czy inżynierii oprogramowania.

\section{Motywacja}
	W przyszłości podłączenie RET-a do komputera (pomijając radio moduł) przy pomocy adaptera \textit{RS-485 -> USB}, w celu skrócenia czasu testowania urządzenia.
