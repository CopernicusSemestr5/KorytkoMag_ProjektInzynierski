\chapter{Podsumowanie}
    Sposób podejscia do implementacji pracy inżynierskiej polegający na zastosowaniu licznych wzorców projektowych sprawił, że powstały kod
    jest modularny oraz elastyczny pod kątem przyszłej rozbudowy czy utrzymania. Powstały produkt posiada wysoką wartość rynkową 
    łatwą do zrealizowania. Zostanie to osiągnięte dzięki implementacji strategii komunikacji, odpowiedzialnej za konwersję ramek HDLC z postaci heksadecymalnej
    do postaci binarnej. Do tego momentu implementacja kolejnych scenariuszy oraz obsługi ramek nadzorujących może być kontynuowana.
    Analiza funkcjonalności wymaganych do zrealizowania poprzez podział na warstwy modelu OSI, umożliwiła zidentyfikować konieczność emulacji warstwy fizycznej. 
    Sprawiło to, że zarówno wykonanie testów modułowych jak i manualnych jest realizowane bez konieczności podłączenia rzeczywistego urządzenia. 
    Podział zadań przy pomocy tablicy Kanban, pozwolił w odpowiednim momencie zaobserwować konieczność wprowadzenia usprawnień 
    w procesie jak i architekturze oprogramowania. Jednoczesne tworzenie diagramu sekwencji oraz klas, pozwaliło postrzegać każdą z funkcjonalności z perspektywy całego systemu.
    Pomimo kilku potknięć, które sprawiły że komunikacja z fizycznym urządzeniem nie została zrealizowana, 

    %%%%%%%%%%%%%%%%%%%%%%%%%%%%%%%%%%%%%%%%%%%%%%%%%%%%%%%%%%%%%%%%%%%%%%%%%%%%%%%%%%%%%%%%%%%%%%%%%%%%%%%%%%%%

    \section{Co zostało zrealizowane}
    Zaimplementowany został sterownik do urządzenia typu RET, który komunikuje sie z użytkownikiem przy pomocy interfejsu konsolowego. Po zestawieniu trzech 
    warstw OSI, wysłano na wybrany port żądanie kalibracji oraz otrzymano odpowiedź na tę wiadomość. Konfiguracja środowiska uruchomieniowego 
    na systemie Manjaro Linux bądź Arch oraz kompilacja plików binarnych, sprowadza się do wywołania zaledwie kilku komend. 
    Wiadomości zapisane są w języku angielskim, zrozumiałym dla człowieka. Ryzyko pojawienia się błędu w trakcie dział
    Osoba korzystająca z programu na bieżąco jest informowana o efektach komunikacji z urządzeniem dzięki rozbudowanemu systemowi logowania, 
    z funkcją zapisu do pliku tekstowego. W przypadku konieczności głębszej analizy działania programu, 
    należy zmienić definicję filtrowania logów na \textit{trace}.
    Tworzenie rozbudowanych wiadomości zapisanych za pomocy wartości heksadecymalnych sprowadzono do podania odpowiedniej komendy
    w języku angielskim oraz adresu portu, do którego podłączony jest symulator urządzenia. 

    %%%%%%%%%%%%%%%%%%%%%%%%%%%%%%%%%%%%%%%%%%%%%%%%%%%%%%%%%%%%%%%%%%%%%%%%%%%%%%%%%%%%%%%%%%%%%%%%%%%%%%%%%%%%

    \section{Co nie zostało zrealizowane}
    Podążając zgodnie z założeniami projektowymi, poniższe wymagania stają się konieczne dopiero w momencie podłączenia rzeczywistego urządzenia
    poprzez port USB. Ważnym krokiem dokonanym, jest dokładne przeanalizowanie problemów, które programista napotka w przyszłości,
    co pozwoli odpowiednio zaplanować fazę ich wdrażania.

    %%%%%%%%%%%%%%%%%%%%%%%%%%%%%%%%%%%%%%%%%%%%%%%%%%%%%%%%%%%%%%%%%%%%%%%%%%%%%%%%%%%%%%%%%%%%%%%%%%%%%%%%%%%%

    \subsection{Walidacja sumy CRC}
    W celu przesłania wiadomości zapisanej w systemie heksadecymalnym przy pomocy interfejsu USB <-> RS-485,
    należy dokonać jej konwersji na postać binarną. Zarówno wiadomość odebrana, jak i wysłana, w trakcie transmisji może ulec wybrakowaniu bądź
    modyfikacji. W tym celu ramka HDLC protokołu AISG 2.0 posiada dwa bajty przeznaczone na walidację takowej wiadomości a jest to wykonywane
    dzięki wyliczeniu sumy CRC-16. W związku z tym, że istnieje wiele implementacji algorytmu wyliczania sumy CRC, a każda z nich może różnić się od siebie zarówno:
    definicją wyrażenia wielomianowego, jego postacią oraz wartością początkową, próbowałem na podstawie ,,podsłuchanej'' wiadomości przy pomocy
    inżynierii wstecznej zdefiniować brakującą wiedzę, lecz aby tego dokonać konieczne było zdobycie informacji na temat zarówno endianowości
    jak i uporządkowania bitów, której na moment opracowywania algorytmu nie posiadano. Pomimo tego, że dla komputera jest to prosta operacja, to weryfikacja przez człowieka
    wartości sumy CRC dla wiadomości o długości 16-tu bajtów na kartce zajmuje naprawdę dużo czasu. W dodatku powszechnym problemem jest to, że 
    producenci urządzeń linii antenowej pomimo tego, że deklarują pełną implementację protokołu AISG 2.0, to w rzeczywistości okazuje się, że
    wcale tak nie jest. Chcąc zapoznać się z protokołem komunikacyjnym oraz zrealizować projekt inżynierski, zamiast walczyć z
    wadliwym urządzeniem, postanowiono utworzyć symulator urządzenia. Pomyślna komunikacja z symulowanym podrzędnych zostaje osiągnięta dzięki przyjęciu stałej wartości
    dla pola przeznaczonego na dwubajtową sumę CRC.
   
    \subsection{Podmiana bajtów o wartości flagi startu/stopu}
    Z języka angielskiego ,,Byte stuffing''. Procedura polega na tym, że zarówno dla flagi startu jak i stopu zarezerwowana jest wartość 0x7E.
    Niestety urządzenie podrzędne inne niż posiadane, może odpowiedzieć na podczas procedur XID negocjacji, wiadomością zawierającą bajt 0x7E. Nieobsłużone
    takie zjawisko spowoduje, że urządzenie nadrzędne zinterpretuje ten bajt jako bajt stopu, co jest niepożądane\cite{ISO-IEC-13239}. W tym celu standard AISG 
    definiuje procedurę do jakiej należy się zastosować w takiej sytuacji, z którą nie zaznajomiłem się.
    
    \subsection{Interwały czasowe}
    Protokół AISG 2.0 definiuje szereg interwałów których obecność można zaobserwować podczas komunikacji z urządzeniem. Jednym z nich jest
    utrzymanie urządzenia w stanie zaadresowanym. W przypadku kiedy urządzenie podrzędne nie otrzyma jakiejkolwiek wiadomości w ciągu 3 minut
    od jego zaadresowania, przechodzi ono w stan niezaadresowany, po czym ponownie należy zestawić warstwę fizyczną oraz łącza danych wraz
    z całą procedurą XID negocjacji. Kolejnymi zależnościami czasowymi są odstępy pomiędzy wysłaniem wiadomości a rozpoczęciem nasłuchiwania na odpowiedź.
    Problemem mógłby być scenariusz w którym wysyłamy wiadomość do urządzenia podrzędnego, a jest ono na przykład w trakcie procedury kalibracji która może 
    trwać ok 2 minuty. Jako odpowiedź możemy wtedy otrzymać ramkę typu S której stan nazwano RNR czyli z angielskiego ,,Receiver Not Ready''. 
    W tej sytuacji należy odczekać pewien kwant czasu, a następnie ponowić komunikację.

    \subsection{Ustalanie maski}
    RET jest urządzeniem które często posiada dodatkowy port RS-485, co oznacza że można podłączyć dwa bądź więcej urządzeń szeregowo. 
    W takiej sytuacji na wiadomość pochodzącą z urządzenia nadrzędnego zaadresowaną wartością 0xFF, każde z nich wyśle swoją odpowiedź, co niesie ze sobą wiele problemów.
    Po stronie urządzenia nadrzędnego to zjawisko zostanie zaobserwowane jako złożenie zawartości wielu wiadomości co będzie skutkowało niepomyślnym
    procesem weryfikacji sumy CRC oraz ilosci flag stopu. Algorytm definiowania maski skutecznie rozwiązuje ten problem. 
    Podczas tej procedury ustala się wartość unikalnego identyfikatora każdego z urządzeń, w celu wysłania wiadomości z żadaniem zaadresowania,
    na którą odpowie tylko jedno urządzenie. Realizacja powyższego problemu często korzysta z algorytmów przeszukiwania binarnego, w trakcie którego
    zadaniem jest ustalenie unikalnego identyfikatora urządzenia podrzędnego. Z racji tego, że podczas pracy inżynierskiej
    przewidywano uruchomienie instancji sterownika dla każdego urządzenia osobno, pominięto implementację wyżej wymienionej logiki.

    \section{Możliwości rozwoju}
    Dzięki elastycznej i skalowalnej architekturze oprogramowania, bez większego problemu powinna być możliwość zaimplementowania rzeczywistej
    warstwy fizycznej. Pozwoli to na podłączenie prawdziwego urządzenia w celu wykonywania zabiegów testowych na nim. Po zrealizowaniu
    tego etapu planowane jest dodanie kolejnych komend definiowanych przez protokół AISG 2.0 takich jak: aktualizacja oprogramowania, 
    reset twardy czy miękki oraz ustawienie kąta. Kolejnym usprawnieniem projektu jest wdrożenie do aplikacji mechanizmu wielowątkowości,
    co pozwoli obsłużyć komunikację z większą liczbą urządzeń z poziomu jednej instancji sterownika.
    Warstwy dynamicznego budowania wiadomości pozwalają również zmniejszyć czas wdrożenia najnowszej wersji protokołu AISG w wersji 3.0.

%czyści puste strony
\let\cleardoublepage\clearpage