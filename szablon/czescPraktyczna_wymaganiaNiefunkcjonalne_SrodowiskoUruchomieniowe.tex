\chapter{Dodatkowe informacje}
\section{Środowisko uruchomieniowe}
	\begin{itemize}
		\item System operacyjny Linux
			\begin{itemize}
				\item Dystrybucje:
				\begin{itemize}
					\item Manjaro 17.01;
				\end{itemize}
				\item Zależności:
					\begin{itemize}
						\item CMake 3.15.4;
						\item Boost C++ Libraries 1.71;
						\item Git -> Pobranie kodu produkcyjnego, GTest, GMock;
						\item GCC 9.2;
					\end{itemize}
			\end{itemize}
	\end{itemize}
\section{Licencje}
	\noindent
	Użyte biblioteki:
	\begin{itemize}
		\item Boost - Boost Software License, licencja typu OpenSource podobna do BSD i MIT;
		\item GTest - BSD 3;
		\item CMake - BSD 3;
		\item ZeroMQ - GNU GPL V3 wraz z koniecznością statycznego linkowania;
	\end{itemize}
	\bigskip
	Kod źródłowy oparty jest na licencji CC BY-NC-ND 4.0 \cite{CC} czyli:
	\begin{itemize}
		\item dzielenie się - kopiuj i rozpowszechniaj utwór w dowolnym medium i formacie;
		\item uznanie autorstwa - należy odpowiednio oznaczyć oraz podać link do recepcji oraz wskazać
		jeśli zostały dokonane w nim zmiany;
		\item użycie niekomercyjne - nie należy wykorzystywać utworu do celów komercyjnych;
		\item bez utworów zależnych - remiksując, przetwarzając lub tworząc na podstawie utworu,
		nie wolno rozpowszechniać zmodyfikowanych treści;
	\end{itemize}
