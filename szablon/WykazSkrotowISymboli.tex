\chapter*{Wykaz skrótów i symboli}
\addcontentsline{toc}{chapter}{Wykaz skrótów i symboli}
\noindent
\textbf{3GPP} --- organizacja normalizująca rozwój telefonii komórkowej;\newline
\textbf{5G} --- standard sieci komórkowej, następca LTE;\newline
\textbf{AISG 2.0} --- protokół bazujący na komunikacji half duplex oraz protokole HDLC;\newline
\textbf{Back-end} --- warstwa oprogramowania obsługująca niskopoziomową logikę biznesową;\newline
\textbf{bajt} --- najmniejsza adresowalna jednostka informacji pamięci komputerowej;\newline
\textbf{bit} --- najmniejsza jednostka informacji w odniesieciu do sprzętu komputerowej;\newline
\textbf{broadcast} --- rozgłoszeniowy tryb transmisji danych;\newline
\textbf{C++} --- język programowania;\newline
\textbf{CRC} --- Cyclic Redundancy Check - system sum kontrolnych;\newline
\textbf{CRC-16} --- 16-bitowy cykliczny kod nadmiarowy;\newline
\textbf{DB} --- database - baza danych;\newline
\textbf{debugowanie} --- proces analizy programu pod kątem zaistniałych błędów;\newline
\textbf{Debug} --- priorytet logowania wskazujący informacje potrzebne podczas debugowania;\newline
\textbf{delete} --- słowo kluczowe języka C++, którego użycie wywołuje destruktor obiektu;\newline
\textbf{driver} --- sterownik - program obsługujący urządzenie podłączone do komputera;\newline
\textbf{enkapsulacja} --- opakowanie danych z wyższej warstwy w warstwie niższej;\newline
\textbf{execute} --- metoda interfejsu klasy ICommand służąca do uruchomienia komendy;\newline
\textbf{Error} --- priorytet logowania informujący o błędnym wykonaniu programu;\newline
\textbf{Front-end} --- warstwa oprogramowania obsługująca odbiór danych od użytkownika;\newline
\textbf{gniazdo} --- dwukierunkowy punkt końcowy połączenia sieciowego;\newline
\textbf{GSM} --- Global System for Mobile Communications - standard telefonii komórkowej;\newline
\textbf{half duplex} --- połączenie w którym naprzemienne jest przesyłanie i odbieranie informacji;\newline
\textbf{HDLC} --- High-Data Link Control - protokół warstwy łącza danych;\newline
\textbf{HDLC body} --- ramka HDLC bez bajtów startu, stopu oraz sumy CRC;\newline
\textbf{h} --- hour - godzina;\newline
\textbf{Info} --- priorytet logowania informujący o ważnych etapach w wykonaniu programu;\newline
\textbf{klasa finalna} --- klasa po której nie można zdefiniować dziedziczenia;\newline
\textbf{little endian} --- cienkokońcowość;\newline
\textbf{LTE} --- Long Term Evolution - standard bezprzewodowego przesyłu danych;\newline
\textbf{ADDR} --- Address - adres docelowy urządzenia budujący ramkę;\newline
\textbf{CTRL} --- Control - bajt kontrolny ramki;\newline
\textbf{GI} --- Group Identifier - identyfikator grupy ramki;\newline
\textbf{GL} --- Group Length - długość groupy czyli liczba bajtów która pojawi się od kolejnej pozycji włącznie aż sumy CRC;\newline
\textbf{PI} --- Parameter Identifier - identyfikator parametru ramki;\newline
\textbf{PL} --- Parameter Length - długość wartości parametru ramki;\newline
\textbf{PL} --- Parameter Length - długość wartości parametru ramki;\newline
\textbf{PV} --- Parameter Value - wartość parametru ramki;\newline

\newpage
\chapter*{Wykaz skrótów i symboli}
\noindent
\textbf{metoda} --- funkcja należąca do klasy;\newline
\textbf{min} --- minute - minuta;\newline
\textbf{model OSI} --- model odniesienia łączenia systemów otwartych;\newline
\textbf{ms} --- milisecond - milisekunda;\newline
\textbf{N(R)} --- receive sequence number, numer porządkowy odebranej ramki;\newline
\textbf{N(S)} --- send sequence number, numer porządkowy wysłanej ramki;\newline
\textbf{null} --- wartość przypisywana do wskaźnika równa 0;\newline
\textbf{numer portu} --- jeden z parametrów gniazda, identyfikujący proces nim zarządzający;\newline
\textbf{OFDMA} --- metoda zwielokrotnienia w dziedzinie częstotliwości;\newline
\textbf{P/F bit} --- Pool/Final bit - obliczany podczas kalkulacji bajtu kontrolnego;\newline
\textbf{payload} --- fragment wiadomości zawierający jedynie istotne informacje;\newline
\textbf{POSIX Regexp} --- standard zapisu wyrażeń regularnych;\newline
\textbf{RAII} --- Resource acquisition is initialization - inicjowanie przy pozyskaniu zasobu;\newline
\textbf{ramka} --- pakiet danych;\newline
\textbf{RET} --- Remote Electrical Tilt - urządzenie zmieniające elektryczny kąt wiązki anteny;\newline
\textbf{RS-485} --- standard transmisji szeregowej;\newline
\textbf{std::any} --- klasa reprezentująca dowolny typ danych;\newline
\textbf{std::string} --- klasa reprezentująca łańcuch znaków;\newline
\textbf{std::vector} --- klasa reprezentująca danymiczną tablicę;\newline
\textbf{STL} --- Standard Template Library - biblioteka C++ w przestrzeni nazw std::;\newline
\textbf{SNRM} --- Set Normal Response Mode - przejdź na synchroniczny tryb komunikacji;\newline
\textbf{s} --- second - sekunda;\newline
\textbf{Trace} --- priorytet logowania najniższej rangi;\newline
\textbf{UI} --- User Interface - interfejs użytkownika;\newline
\textbf{UMTS} --- Universal Mobile Telecommunications System - standard telefonii komórkowej;\newline
\textbf{UniqueId} --- złożenie numeru seryjnego wraz z kodem producenta;\newline
\textbf{USB} --- Universal Serial Bus - uniwersalna magistrala szeregowa;\newline
\textbf{Warning} --- priorytet logowania informujący o wykonaniu mogącym powodować błędy;\newline
\textbf{WCDMA} --- technika związana z dostępem do sieci radiowej;\newline
\textbf{UA} --- Unnumbered Acknowledged - zaakceptowana ramka nienumerowana;\newline
