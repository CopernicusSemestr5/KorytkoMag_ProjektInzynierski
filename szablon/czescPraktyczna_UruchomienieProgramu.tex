\chapter{Uruchomienie programu}
    \section{Konfiguracja środowiska}
    Napisy ujęte w cudzysłowy są komendami które należy wywołać z lini poleceń.
    \begin{enumerate}
        \item Uruchomienie systemu Manjaro Linux;
        \item Podłączenie do internetu w celu pobrania repozytorium;
        \item Konfiguracja środowiska (pomiń w przypadku posiadania kompletu):
        \begin{enumerate}
            \item ,,sudo pacman --sync boost''
            \item ,,sudo pacman --sync cmake''
            \item ,,sudo pacman --sync git''
        \end{enumerate}
    \end{enumerate}
    \section{Kompilacja kodu źródłowego oraz wystartowanie programu}
    \begin{enumerate}
        \item otwarcie pierwszej konsoli w celu uruchomienia symulatora urządzenia:
		\begin{enumerate}
			\item ,,git clone -{}-recursive https://github.com/trunksBT/KorytkoMag\_RetSimulator.git''
			\item ,,cd KorytkoMag\_RetSimulator''
			\item ,,cmake .''
			\item ,,bash runBinary.sh''
		\end{enumerate}
		\item otwarcie drugiej konsoli w celu uruchomienia sterownika:
		\begin{enumerate}
			\item ,,git clone -{}-recursive https://github.com/trunksBT/KorytkoMag\_RetDriverSimulator.git''
			\item ,,cd KorytkoMag\_RetDriverSimulator''
			\item ,,cmake .''
			\item ,,bash runBinary.sh''
			\item Wywołanie komend zgodnie z diagramem sekwencji zatytułowanym ,,KalibracjaRETa''
			\item Aby zakończyć trzeba wpisać ,,exit''
		\end{enumerate}
    \end{enumerate}
    \newpage
    \section{Efekt końcowy}
        \lstinputlisting[
            language=bash,
            basicstyle=\tiny,
            caption=Wykonanie programu,
            label=lst:WykonanieProgramu]
        {LogZWykonania/BinaryLog.log}
