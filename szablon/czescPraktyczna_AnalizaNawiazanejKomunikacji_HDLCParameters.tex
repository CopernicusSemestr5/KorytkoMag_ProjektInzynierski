\section{Negocjacje parametrów HDLC}
Cechą negocjacji parametrów przy pomocy ramek XID jest to, że jeśli żądana wartość jest wspierana przez
urządzenie podrzędne, to odpowie ono wiadomością zawierającą te same parametry oraz te same wartości. 
W przeciwnym wypadku otrzymanymi wartościami parametrów będą największe możliwe przez nie wspierane.
Zaobserwowano to zjawisko w przypadku negocjacji wielkości payloadu dla wysłanej oraz otrzymanej ramki informacyjnej.

Analiza wartości wiadomości wychodzącej 
(rysunek \ref{lst:WykonanieProgramu}, linijki {30, 31} oraz \ref{fig:DiagramSequence_HDLCParameters_SNRM}):
\begin{enumerate}
    \item ADDR = 0x03 --- adres nadany urządzeniu podrzędnemu;
    \item CTRL = 0xBF --- charakterystyczny dla ramki XID;
    \item FI = 0x81 --- charakterystyczna dla ramki XID oraz grupy wiadomości przypisania adresu;
    \item GI = 0xF0;
    \item GL = 0x12;
    \item Pierwszy parametr HDLC:
    \begin{enumerate}
        \item PI = 0x05 --- maksymalny rozmiar payloadu wysyłanej ramki I;
        \item PL = 0x04;
        \item PV = \{0xF0, 0x2D, 0x00, 0x00\} czyli 61485 bity;
    \end{enumerate}
    \item Drugi parametr HDLC:
    \begin{enumerate}
        \item PI = 0x06 --- maksymalny rozmiar payloadu odbieranej ramki I;
        \item PL = 0x04;
        \item PV = \{0xF0, 0x2D, 0x00, 0x00\} czyli 61485 bity;
    \end{enumerate}
    \item Trzeci parametr HDLC:
    \begin{enumerate}
        \item PI = 0x07 --- maksymalna liczba ramek wysłanych;
        \item PL = 0x01;
        \item PV = 0x01;
    \end{enumerate}
    \item Czwarty parametr HDLC:
    \begin{enumerate}
        \item PI = 0x08 --- maksymalna liczba ramek odebranych;
        \item PL = 0x01;
        \item PV = 0x01;
    \end{enumerate}
\end{enumerate}

Analiza wartości wiadomości przychodzącej 
(rysunek \ref{lst:WykonanieProgramu}- linijka 32, rysunek \ref{fig:DiagramSequence_HDLCParameters_SNRM}):
\begin{enumerate}
    \item ADDR = 0x03 --- urządzenie identyfikuje się żądanym adresem;
    \item CTRL = 0xBF --- charakterystyczny dla ramki XID;
    \item FI = 0x81 --- charakterystyczna dla ramki XID oraz grupy wiadomości przypisania adresu;
    \item GI = 0x80;
    \item GL = 0x12;
    \item Pierwszy parametr HDLC:
    \begin{enumerate}
        \item PI = 0x05 --- maksymalny rozmiar payloadu wysyłanej ramki I;
        \item PL = 0x04;
        \item PV = \{0x50, 0x02, 0x00, 0x00\} czyli 20482 bity
    \end{enumerate}
    \item Drugi parametr HDLC:
    \begin{enumerate}
        \item PI = 0x06 --- maksymalny rozmiar payloadu odbieranej ramki I;
        \item PL = 0x04;
        \item PV = \{0x50, 0x02, 0x00, 0x00\} czyli 20482 bity
    \end{enumerate}
    \item Trzeci parametr HDLC:
    \begin{enumerate}
        \item PI = 0x07 --- maksymalna liczba ramek wysłanych;
        \item PL = 0x01;
        \item PV = 0x01;
    \end{enumerate}
    \item Czwarty parametr HDLC:
    \begin{enumerate}
        \item PI = 0x08 --- maksymalna liczba ramek odebranych;
        \item PL = 0x01;
        \item PV = 0x01;
    \end{enumerate}
\end{enumerate}
