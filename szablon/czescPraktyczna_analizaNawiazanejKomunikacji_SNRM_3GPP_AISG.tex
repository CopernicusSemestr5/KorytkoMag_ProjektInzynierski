\section{Przejście na normalny tryb komunikacji}
Analiza wartości wiadomości wychodzącej 
(rysunek \ref{lst:WykonanieProgramu}, linijki 35, 36 oraz \ref{fig:DiagramSequence_HDLCParameters_SNRM}):
\begin{enumerate}
    \item ADDR = 0x03 --- adres nadany urządzeniu podrzędnemu;
    \item CTRL = 0x93 --- charakterystyczny dla ramki U, SNRM;
\end{enumerate}

Analiza wartości wiadomości przychodzącej
(rysunek \ref{lst:WykonanieProgramu}, linijka 37 oraz \ref{fig:DiagramSequence_HDLCParameters_SNRM}):
\begin{enumerate}
    \item ADDR = 0x03 --- adres nadany urządzeniu podrzędnemu;
    \item CTRL = 0x73 --- charakterystyczny dla ramki U, UA;
\end{enumerate}

\section{Wersja standardu 3GPP}
Ciekawą obserwacją tej wiadomości jest fakt, że identyfikator parametru równy 0x05 pojawił się podczas definiowania długości payloadu
dla ramki informacyjnej. Różnica polega na tym, że tamta wiadomość posiadała identyfikator grupy równy 0x80

Analiza wartości wiadomości wychodzącej 
(rysunek \ref{lst:WykonanieProgramu}, linijki {40, 41} oraz \ref{fig:DiagramSequence_3GPP_AISGVersion_Calibrate}):
\begin{enumerate}
    \item ADDR = 0x03 --- adres nadany urządzeniu podrzędnemu;
    \item CTRL = 0xBF --- charakterystyczny dla ramki XID;
    \item FI = 0x81 --- charakterystyczna dla ramki XID oraz grupy wiadomości przypisania adresu;
    \item GI = 0xF0;
    \item GL = 0x03;
    \item Parametr HDLC:
    \begin{enumerate}
        \item PI = 0x05 --- wersja standardu 3GPP;
        \item PL = 0x01;
        \item PV = 0x08 --- pionierska dla technologii LTE
    \end{enumerate}
\end{enumerate}

Analiza wartości wiadomości przychodzącej jest zbędna gdyż posiada te same wartości.


\section{Wersja protokołu AISG}
Analiza wartości wiadomości wychodzącej 
(rysunek \ref{lst:WykonanieProgramu}, linijki {45, 46} oraz \ref{fig:DiagramSequence_3GPP_AISGVersion_Calibrate}):
\begin{enumerate}
    \item ADDR = 0x03 --- adres nadany urządzeniu podrzędnemu;
    \item CTRL = 0xBF --- charakterystyczny dla ramki XID;
    \item FI = 0x81 --- charakterystyczna dla ramki XID oraz grupy wiadomości przypisania adresu;
    \item GI = 0xF0;
    \item GL = 0x03;
    \item Parametr HDLC:
    \begin{enumerate}
        \item PI = 0x14 --- wersja protokołu AISG;
        \item PL = 0x01;
        \item PV = 0x02 --- wybrano wersję v2.0, istnieją jeszcze: 1.0, 1.1 oraz 3.0
    \end{enumerate}
\end{enumerate}
Analiza wartości wiadomości przychodzącej jest zbędna gdyż posiada te same wartości.
