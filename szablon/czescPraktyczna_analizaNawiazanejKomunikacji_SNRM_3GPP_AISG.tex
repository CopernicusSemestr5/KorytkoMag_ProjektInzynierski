\section{Przejście na normalny tryb komunikacji}
Jako odpowiedź na tę wiadomość urządzenie otrzyma ramkę, która zawiera wartość 0x73, co oznacza że 
potwierdza ono żądane oczekiwanie.
\newline\newline
Bajt kontrolny posiada wartość 0x93 i oznacza żadanie ustalenia sposobu komunikacji z urządzeniem podrzędnym na tryb normalny, 
a więc taki w którym wysyła ono ramkę jedynie jako odpowiedź na wcześniej nadaną przez urządzenie nadrzędne.

\section{Wersja standardu 3GPP}
Jako odpowiedź na tę wiadomość urządzenie otrzyma ramkę, która zawiera wartość 0x73, co oznacza że 
potwierdza ono żądane oczekiwanie.
\newline\newline
Bajt kontrolny posiada wartość 0x93 i oznacza żadanie ustalenia sposobu komunikacji z urządzeniem podrzędnym na tryb normalny, 
a więc taki w którym wysyła ono ramkę jedynie jako odpowiedź na wcześniej nadaną przez urządzenie nadrzędne.

Teraz można przejść do coraz bardziej szczegółowych parametrów jak ustanowienie numeru wersji standardu 3GPP.
Podczas tworzenia tej wiadomości znów skorzystano z ramki XID a więc należy zdefiniować identyfikator formatu, grupy oraz jej długość.
Pierwsze dwie wartości są równe tym z komendy ,,AddressAssignment'' a długość ma wartość 0x03.
Identyfikatorem parametru jest tym razem wartość 0x05, długością 0x01 natomiast wartościa parametru 0x08.
Jest to wersja standardu w wersji 8-mej czyli pionierska dla technologii \textit{LTE}.
Wprowadzono w niej wparcie dla interfejsu radiowego opartego o \textit{OFDMA}, którego wykorzystanie można zaobserwować również przy technologii 5G.

\section{Wersja protokołu AISG}
Z racji tego, że implementowana wersja protokołu AISG 2.0 jest jedną z wielu ( istnieją jeszcze dwie wcześniejsze: 1.0, 1.1 oraz najnowsza 3.0),
potrzeba zdefiniować według którego standardu budowane są przez urządzenie nadrzędne oraz rozpoznawane przez podrzędne.
Bajt identyfikujący parametr dla tej wiadomości ma wartość 0x14, długość parametru to 0x01 a jej wartość to 0x02.