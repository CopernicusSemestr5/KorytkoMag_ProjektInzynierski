\chapter{Plan dalszego rozwoju}
\section{Funkcjonalności niezrealizowane}
    W celu przesłania wiadomości zapisanej w systemie heksadecymalnym przy pomocy interfejsu USB <-> RS-485
    należy dokonać konwersji na postać binarną. Zarówno wiadomość odebrana, jak i wysłana może w trakcie transmisji ulec wybrakowaniu bądź
    zmianie zawartości. W tym celu ramka HDLC protokołu AISG 2.0 posiada dwa bajty przeznaczone na walidację takowej wiadomości a jest to wykonywane
    dzięki wyliczeniu sumy CRC-16. W związku z tym, że istnieje wiele implementacji algorytmu wyliczania sumy CRC, a każda z nich może różnić się od siebie zarówno:
    definicją wyrażenia wielomianowego, jego postacią oraz wartością początkową, próbowałem na podstawie ,,podsłuchanej'' wiadomości przy pomocy
    inżynierii wstecznej zdefiniować brakującą wiedzę, lecz aby tego dokonać potrzebowałem więcej informacji na temat zarówno endianowości
    jak i uporządkowania bitów, których na moment opracowywania algorytmu nie posiadałem. Pomimo tego, że dla komputera jest to prosta operacja, to weryfikacja przez człowieka
    wartości sumy CRC dla wiadomości o długości 16-tu bajtów na kartce zajmuje naprawdę dużo czasu. W dodatku powszechnym problemem jest to, że 
    producenci urządzeń linii antenowej pomimo tego, że deklarują pełną implementację protokołu AISG 2.0, to w rzeczywistości okazuje się, że
    wcale tak nie jest. Chcąc zapoznać się z protokołem komunikacyjnym oraz zrealizować projekt inżynierski podczas którego sprawdzę umiejętności testowania,
    wdrażania wzorców projektowych czy też posługiwania się językiem C++, zamiast walczyć z urządzeniem i jego możliwymi błędami, postanowiłem
    utworzyć symulator urządzenia, z którym sterownik będzie komunikował się przy pomocy biblioteki ZeroMQ, a samą wartość sumy wyznaczyłem na \{0x13, 0x37\}. 
    Dzięki zaznajomieniu się z wieloma wzorcami projektowymi jak fabryka, komenda, budowniczy czy nakładanie obostrzeń na program dzięki listowaniu obsługiwanych komend, 
    wspomniany wcześniej symulator urządzenia powstał niemalże za darmo, co uznaję za bardzo duży sukces z dziedziny projektowania architektury oprogramowania.
    Kolejną zaletą utworzenia symulatora urządzenia jest umożliwienie testów komponentowych realizując regułę czarnej skrzynki, co pozwoli w przyszłości
    znacznie skrócić czas testowania sterownika pod względem błędów logicznych.

\section{Rozbudowa aplikacji o nowe funkcjonalności}
